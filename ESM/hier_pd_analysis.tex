\documentclass[american,floatsintext,doc]{apa6}

\usepackage{amssymb,amsmath}
\usepackage{ifxetex,ifluatex}
\usepackage{fixltx2e} % provides \textsubscript
\ifnum 0\ifxetex 1\fi\ifluatex 1\fi=0 % if pdftex
  \usepackage[T1]{fontenc}
  \usepackage[utf8]{inputenc}
\else % if luatex or xelatex
  \ifxetex
    \usepackage{mathspec}
    \usepackage{xltxtra,xunicode}
  \else
    \usepackage{fontspec}
  \fi
  \defaultfontfeatures{Mapping=tex-text,Scale=MatchLowercase}
  \newcommand{\euro}{€}
\fi
% use upquote if available, for straight quotes in verbatim environments
\IfFileExists{upquote.sty}{\usepackage{upquote}}{}
% use microtype if available
\IfFileExists{microtype.sty}{\usepackage{microtype}}{}

% Table formatting
\usepackage{longtable,booktabs}
\usepackage[counterclockwise]{rotating}   % Landscape page setup for large tables
\usepackage{multirow}		% Table styling
\usepackage{tabularx}		% Control Column width
\usepackage[flushleft]{threeparttable}	% Allows for three part tables with a specified notes section
\usepackage{threeparttablex}            % Lets threeparttable work with longtable
\usepackage{longtable}              % Allows tables to break across pages

\ifxetex
  \usepackage[setpagesize=false, % page size defined by xetex
              unicode=false, % unicode breaks when used with xetex
              xetex]{hyperref}
\else
  \usepackage[unicode=true]{hyperref}
\fi
\hypersetup{breaklinks=true,
            pdfauthor={},
            pdftitle={Supplementary materials for the journal article entitled Distorted estimates of implicit and explicit learning in applications of the process-dissociation procedure to the SRT task},
            colorlinks=true,
            citecolor=blue,
            urlcolor=blue,
            linkcolor=black,
            pdfborder={0 0 0}}
\urlstyle{same}  % don't use monospace font for urls

\setlength{\parindent}{0pt}
%\setlength{\parskip}{0pt plus 0pt minus 0pt}

\setlength{\emergencystretch}{3em}  % prevent overfull lines

\setcounter{secnumdepth}{0}
\ifxetex
  \usepackage{polyglossia}
  \setmainlanguage{}
\else
  \usepackage[american]{babel}
\fi

% Manuscript styling
\captionsetup{font=singlespacing,justification=justified}
\usepackage{csquotes}

 % Line numbering
  \usepackage{lineno}
  \linenumbers


\usepackage{tikz} % Variable definition to generate author note

% Essential manuscript parts
  \title{Supplementary materials for the journal article entitled
\enquote{Distorted estimates of implicit and explicit learning in
applications of the process-dissociation procedure to the SRT task}}

  \shorttitle{Hierarchical PD}


  \author{
          Christoph Stahl,
          Marius Barth,
          \& Hilde Haider  }

  \def\affdep{{"", "", ""}}%
  \def\affcity{{"", "", ""}}%

  \affiliation{
    \vspace{0.5cm}
          \textsuperscript{} University of Cologne  }


  \def\affinst{{"init", "University of Cologne"}}%
  \def\affstate{{"init", ""}}%
  \def\affcntry{{"init", ""}}%

  \note{
    \vspace{1cm}
    Author note

    \raggedright
    \setlength{\parindent}{36pt}

    \newcounter{author}

%     %     %       %     ;
%     %       %     ;
%     %       %     .
%     
    This work was funded by Deutsche Forschungsgemeinschaft grants
    STA-1269/1-1 and HA-5447/8-1.

                                            }

  
  \usepackage{lscape}

\begin{document}

\maketitle



We analysed our data using a modified version of Rouder, Lu, Morey, Sun,
\& Speckman (2008)'s three-level hierarchical process-dissociation
model.

The first level is the process-dissociation model:

\[ I_{ijk} = C_{ijk} + (1-C_{ijk}) A_{ijk}\]

and

\[ E_{ijk} = (1-C_{ijk}) A_{ijk}\]

where \(i\) and \(j\) index participants and items, and \(k\) indexes
the experimental condition. The parameters \(A\) and \(C\) represent
probabilities that range between zero and one; they are transformed via
a probit link to the reals, where \(a\) and \(c\) denote the transformed
parameters:

\(A_{ijk} = \Phi(a_{ijk})\) and \(C_{ijk} = \Phi(c_{ijk})\)

The second level is a main effects models on transformed parameters
\(a\) and \(c\):

\[ c_{ijk} = \alpha_i^{(c)} + \beta_j^{(c)} + \mu_k^{(c)}\]

and

\[ a_{ijk} = \alpha_i^{(a)} + \beta_j^{(a)} + \mu_k^{(a)}\]

where \(\alpha\) denotes participant effects, \(\beta\) denotes item
effects, and \(\mu\) denotes condition effects that lead to conscious or
unconscious contributions to task performance.

Participant and item effects are modeled as draws from bivariate normals
whose covariance matrices were estimated from the data:

\(\left({\alpha_i^{(c)}}\over{\alpha_i^{(a)}} \right) \thicksim N_2(0, \sum_{\alpha}), i=1, \dotsb, I.\)

and

\(\left({\beta_i^{(c)}}\over{\beta_i^{(a)}} \right) \thicksim N_2(0, \sum_{\beta}), j=1, \dotsb, J.\)

This model was estimated within a Bayesian modeling framework using MCMC
sampling. For further detail, refer to Rouder et al. (2008).

\section{Results}\label{results}

For each group, we sampled three chains of 50,000 iterations, discarding
the first 20,000 as burn-in. Mixing was monitored by \(\hat{R}\) which
was below 1.2. Table 1 shows estimates of the posterior distribution of
the grand-mean parameters \(\mu_k\) of the model. Table 2 shows the
estimates equivalent to \(C\) and \(A\) from traditional analyses. As
can be seen, the results corroborated the findings obtained with the
traditional analyses reported above (i.e., \(C>0\), \(A>.2\), and the
ordering of \(A\) estimates across conditions).

\begin{landscape}

\begin{table}[tbp]
\centering
\caption{Parameter estimates from the hierarchical process-dissociation model. Parameters $\mu_k^{(c)}$ and $\mu_k^{(a)}$ refer to estimates of the grand-mean}
\begin{center}
\begin{threeparttable}
\begin{tabular}{llllllllll}
\toprule
\multicolumn{2}{l}{} & \multicolumn{4}{c}{Full dataset} & \multicolumn{4}{c}{Reversals excluded}\\
\multicolumn{2}{l}{} & \multicolumn{2}{c}{$\mu_k^{(a)}$} & \multicolumn{2}{c}{$\mu_k^{(c)}$} & \multicolumn{2}{c}{$\mu_k^{(a)}$} & \multicolumn{2}{c}{$\mu_k^{(c)}$}\\
\midrule
No-learning & Free & -0.82 & -0.96, -0.67 & -6.11 & [-8.61, -4.18] & -0.75 & -0.90, -0.61 & -8.58 & -12.95, -5.51\\
No-learning & Cued & -0.82 & -1.02, -0.62 & -5.13 & [-7.38, -3.44] & -0.74 & -0.94, -0.55 & -5.88 & -7.97, -4.11\\
Permuted & Free & -0.71 & -0.87, -0.55 & -6.30 & [-9.92, -3.70] & -0.64 & -0.79, -0.48 & -7.48 & -10.98, -4.61\\
Permuted & Cued & -0.88 & -1.04, -0.71 & -6.62 & [-10.59, -3.62] & -0.77 & -0.94, -0.61 & -7.14 & -9.56, -4.64\\
Random & Free & -0.88 & -1.04, -0.73 & -6.40 & [-12.02, -3.66] & -0.78 & -0.94, -0.62 & -6.88 & -11.43, -3.61\\
Random & Cued & -0.78 & -0.95, -0.61 & -4.06 & [-5.79, -2.69] & -0.68 & -0.85, -0.52 & -4.39 & -6.12, -2.86\\
\bottomrule
\end{tabular}
\tablenotes{\textit{Note.} 95\% credible intervals are in parentheses.}
\end{threeparttable}
\end{center}
\end{table}

\end{landscape}

\clearpage

\begin{table}[tbp]
\centering
\caption{Parameter estimates from the hierarchical PD model. Parameters $A$ and $C$ denote the Bayesian equivalent to parameter estimates obtained from classical analyses}
\begin{center}
\begin{threeparttable}
\begin{tabular}{llllllllll}
\toprule
\multicolumn{2}{l}{} & \multicolumn{4}{c}{Full dataset} & \multicolumn{4}{c}{Reversals excluded}\\
\multicolumn{2}{l}{} & \multicolumn{2}{c}{$A$} & \multicolumn{2}{c}{$C$} & \multicolumn{2}{c}{$A$} & \multicolumn{2}{c}{$C$}\\
\midrule
No-learning & Free & .21 & [.21, .22] & .03 & [.03, .04] & .23 & [.22, .24] & .03 & [.03, .04]\\
No-learning & Cued & .23 & [.22, .24] & .04 & [.03, .05] & .26 & [.24, .27] & .04 & [.03, .05]\\
Permuted & Free & .25 & [.23, .26] & .03 & [.03, .04] & .27 & [.26, .28] & .03 & [.02, .03]\\
Permuted & Cued & .20 & [.19, .21] & .04 & [.03, .05] & .23 & [.22, .24] & .04 & [.03, .04]\\
Random & Free & .20 & [.19, .21] & .02 & [.02, .03] & .22 & [.21, .23] & .04 & [.03, .04]\\
Random & Cued & .22 & [.21, .23] & .03 & [.02, .03] & .25 & [.24, .26] & .03 & [.02, .04]\\
\bottomrule
\end{tabular}
\end{threeparttable}
\tablenotes{\textit{Note.} 95\% credible intervals are in parentheses.}
\end{center}
\end{table}

\section*{References}\label{references}
\addcontentsline{toc}{section}{References}

Rouder, J. N., Lu, J., Morey, R. D., Sun, D., \& Speckman, P. L. (2008).
A hierarchical process-dissociation model. \emph{Journal of Experimental
Psychology: General}, \emph{137}(2), 370--389.
doi:\href{http://dx.doi.org/10.1037/0096-3445.137.2.370}{10.1037/0096-3445.137.2.370}



\end{document}
